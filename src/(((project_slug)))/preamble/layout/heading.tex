\usepackage[newparttoc, clearempty]{titlesec}

%% Chapter without number, but included in header and TOC
\NewDocumentCommand{\chapternotnumbered}{m}{
  \chapter*{#1}
  \markboth{#1}{#1}
  \addcontentsline{toc}{chapter}{#1}
}

%% Title formatting (section, subsection, paragraph, part)
\titleformat{\section}
  {\MakeLinkTarget[section]{}\Large\sffamily\bfseries}
  {\thesection}
  {0.8em}{}
\titleformat{\subsection}
  {\MakeLinkTarget[subsection]{}\large\sffamily\bfseries}
  {\thesubsection}
  {0.8em}{}
\titleformat{\paragraph}[runin]
  {\MakeLinkTarget[paragraph]{}\normalsize\sffamily\bfseries}
  {\theparagraph}
  {0em}{}
\titleformat{\part}[display]
  {\thispagestyle{empty}\sffamily}
  {\LARGE \partname~\Romanbar{\thepart}}
  {0.2em}{\fontsize{30pt}{36pt}\selectfont\bfseries}

%% Format the chapter heading letter
%%   #1 - chapter number or first letter of chapter title
\NewDocumentCommand{\chapterheadingletter}{m}{%
  \makebox[0pt][l]{%
    \raisebox{-16pt}[0pt][0pt]{%
      \hspace{0.55em}\color{ChapterNumberColor}%
      \usefont{T1}{qzc}{m}{it}\fontsize{95pt}{95pt}\selectfont%
      #1%
    }%
  }%
}

%% Format the chapter heading
%%   #1 - star (*) if unnumbered chapter
%%   #2 - chapter numbering
%%   #3 - chapter title
\NewDocumentCommand{\chaphead}{s o m}{
  \vspace*{-25pt}% header top margin
  {%
    \setlength{\parindent}{0pt}\raggedright%
    \Huge\sffamily\bfseries%
    ((* if "fancy_header_footer" in layout_options *))
    \IfValueT{#2}{%
      \chapterheadingletter{#2}%
    }%
    ((* else *))
    \IfBooleanF{#1}{#2\hspace{0.7em}}%
    ((* endif *))
    #3\par\nobreak%
    \vspace{20pt}% header bottom margin
  }%
}

%% Title formatting (chapter)
\makeatletter
% for numbered Chapters simply use their number
\def\@makechapterhead#1{\chaphead[\thechapter]{#1}}
((* if language == "Chinese" *))
% do not use the first letter for unnumbered Chapters in Chinese
\def\@makeschapterhead#1{\chaphead*{#1}}
((* else *))
% extract first letter of the unnumbered Chapter title
\def\@makeschapterhead#1{\chaphead*[\ExtractLeadingChars{#1}]{#1}}
((* endif *))
\makeatother

%% Extract leading characters from a string
%%   #1 - number of leading characters to extract (default: 1)
%%   #2 - input string
\ExplSyntaxOn
\cs_generate_variant:Nn \tl_item:nn { f }
\NewDocumentCommand {\ExtractLeadingChars} {O{1} m}
  { \tl_item:fn {#2} {#1} }
\ExplSyntaxOff
