\usepackage{keytheorems}
\usepackage{tcolorbox}

\tcbuselibrary{skins, breakable, theorems}

%% Default style options for `tcolorbox` environments
% Save depth of character "j" (typical descender)
\newlength{\depthofj} \settodepth{\depthofj}{j}
\tcbset{
  skin=enhanced,
  colframe=black, colback=white,
  boxrule=0.4pt, arc=3pt, boxsep=0em,
  left=0.9em, right=0.9em, top=1.3ex, bottom=1.3ex-0.8\depthofj,
  % enlarge bottom at break by=-\depthofj,
  % pad at break*=3ex,
  beforeafter skip=0.5\baselineskip plus 2pt,
}
% Persist font color from the text before the box
% see https://github.com/T-F-S/tcolorbox/issues/305
\tcbset{every box/.style={coltext=.}}

%% Definitions of theorem/remark-like environments
\newkeytheoremstyle{thmcommon}{
  headfont=\bfseries\sffamily,
  notefont=\mdseries\sffamily,
  bodyfont=\normalfont,
  headpunct={\strut.},
}

%% More flexible placement of "qed" symbol
\NewCommandCopy{\qedhereold}{\qedhere}
% #1 - star (*) for shift corresponding to big operators
% #2 - optional argument for custom shift length
\RenewDocumentCommand{\qedhere}{s o}{%
  \def\qedshift{0.6ex}%
  \IfBooleanT{#1}{\def\qedshift{1.5ex}}%
  \IfValueT{#2}{\def\qedshift{#2}}%
  \NewCommandCopy{\qedsymbolold}{\qedsymbol}%
  \renewcommand*{\qedsymbol}{\raisebox{-\qedshift}{\qedsymbolold}}%
  \qedhereold%
}
